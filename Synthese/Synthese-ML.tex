\documentclass[a4paper,11pt]{article}
\usepackage[utf8]{inputenc}
\usepackage[T1]{fontenc}
\usepackage[french]{babel}
\usepackage{textcomp}
\usepackage{listings}
\usepackage{pdfpages}
\usepackage{array}
\usepackage{titling}
\usepackage{geometry}

\geometry{hmargin=2.5cm,vmargin=1.5cm}
\setlength{\hoffset}{-18pt}        
\setlength{\oddsidemargin}{0pt} % Marge gauche sur pages impaires
\setlength{\evensidemargin}{9pt} % Marge gauche sur pages paires
\setlength{\marginparwidth}{54pt} % Largeur de note dans la marge
\setlength{\textwidth}{481pt} % Largeur de la zone de texte (17cm)
\setlength{\marginparsep}{2pt} % Séparation de la marge
\setlength{\topmargin}{0pt} % Pas de marge en haut
\setlength{\headheight}{10pt} % Haut de page
\setlength{\headsep}{10pt} % Entre le haut de page et le texte
\setlength{\footskip}{27pt} % Bas de page + séparation
\setlength{\textheight}{708pt} % Hauteur de la zone de texte (25cm)

\setlength{\droptitle}{-4cm}
\title{Synthèse - Machine Learning}
\author{Léa Calem - Fatima Layla - Laureline Martin}

\begin{document}

\maketitle
	
\section{Description du jeu de données}
	\subsection{Quel est le type de problème ?}
		Dans ce projet, nous devons classifier des images. les images reprensentent soit des t-shirt soit des robes.\\
		Le problème est de type supervisé car nos données sont déjà annotées :
		$$ S = {(x_i, y_i)} $$
		Tel que : $x_i$ = ensemble des images, 
		$y_i$ = ensemble des étiquettes des classes $C_1$ et $C_2$.\\
		Classification : $y_i = {C_1, C_2}$.
		Avec les classes : $C_1 =$ \{0 T-shirt/top\} et $C_2 =$ \{3 Dress\}.\\
		\\
		À l’aide des méthodes d’apprentissage, on recherche la fonction $h(x)$ qui à toutes images $x$ associées une étiquette $\hat(y)$. Le but est d’obtenir des valeurs $\hat(y)_i$ proches des $y_i$, pour tout $(x_i, y_i)$ appartenant à $S$.\\
		(cours 02\_methodo\_etu.pdf/ slide 9)

	\subsection{quelles sont les données ?}
		\subsubsection{Modélisation des données sous forme de matrice}
			Nombre d’observations : 14 000 :
			\begin{itemize}
				\item 7 000 de la classe $C_1$
				\item 7 000 de la classe $C_2$
			\end{itemize}
		\subsubsection{Séparation des jeux de données}
			\begin{enumerate}
				\item Données d’entraînement : sous-ensemble de données destiné à l’apprentissage du modèle. On utilise 75\% des données, soit 10 500 .
				\item Données de test : sous-ensemble de données destiné à l’évaluation du modèle (ce jeu de données ne doit en aucun cas être utilisé lors de la conception du modèle). On utilise 25\% des données, soit 3 500.
			\end{enumerate}
	
	\subsection{Description des données}
		Les données sont des images de taille 28x28 pixels (784 pixels) composées de niveau de gris (valeur allant de 0 à 255). Sur ces images, seul l’objet est coloré donc le reste de l’image est en blanc, la valeur des pixels à 0.

	\subsection{Description statistique}
		Graphe du nombre de pixels différents de 0 en fonction de la classe.
		On divise l’image en 3, on cherche la moyenne du blanc de la 1ère et 3éme partie.

	\subsection{Paramètres}
		Étant donné que les images $x_i$ sont bruitées, on suppose que les pixels ayant une valeur < 25 sont blanc.\\Comme paramètre : écart entre blanc et gris tous les 28 pixels.


\newpage
\section{Méthodologie}
	\subsection{Méthodes d'apprentissage utilisées}
		\begin{itemize}
			\item Les K-NN (K plus proches voisins) : Fatima
			\item La régression : Lauréline
			\item SVM : Léa
		\end{itemize}

	\subsection{Protocole de comparaison}
		Pour comparer les résultats des différentes méthodes d'apprentissages utilisées, nous évaluons le taux d’erreurs sur des jeux identiques de données.


\end{document}